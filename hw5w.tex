\documentclass[12pt,letterpaper,boxed]{hmcpset}

\usepackage[margin=1in]{geometry}

\usepackage{graphicx}

\name{Daniel Bashir}
\class{Math 171}
\assignment{Homework 5w}
\duedate{02/14/2018}

\begin{document}

\problemlist{2.6.24, 2.6.33, 3.3.3, 4.4.6, 4.4.10}

\begin{problem}[2.6.24]
$\textbf{Groups with only trivial subgroups.}$ The finite group $G$ has more than one element and no non-trivial subgroups. Prove that $G$ is cyclic of order $p$, where $p$ is a prime number. 
\end{problem}

\begin{solution}
\end{solution}

\clearpage

\begin{problem}[2.6.33]
$\textbf{When is the product of two subgroups a subgroup?}$ Assume that $H$ and $K$ are subgroups of a group $G$. Show that $HK$ is a group if and only if $HK = KH$.
\end{problem}

\begin{solution}

\end{solution}

\clearpage

\begin{problem}[3.3.3]
How many elements of $S_{100}$ have exactly one cycle of size 60? If you randomly pick an element of $S_{100}$, what is the probability that the element will have exactly one cycle of size 60?
\end{problem}

\begin{solution}
\end{solution}

\clearpage


\begin{problem}[4.4.6]
Give a one sentence proof that the conjugacy classes of a group partition the group. Then find all the conjugacy classes of $D_8$. 
\end{problem}

\begin{solution}
\end{solution}

\clearpage

\begin{problem}[4.4.10]
Let $G = \{a,b,c,d,e,f\}$ and let $\Omega = \{x,y,z,u,v,w\}$. We know that $G$ is a group and $\Omega$ is a set. We also know that $G$ acts on $\Omega$. The following table tells us how every element of $G$ acts on elements of $\Omega$. 
\vspace{40mm}
So for example, $c \cdot x = w$ and $c \cdot u = u$. 
\begin{itemize}
  \item[(\textit{a})] What is $(bc) \cdot y$?
  
  \item[(\textit{b})] Can you find a subgroup of $G$ with one element? What about a subgroup of $G$ with two elements? What about a subgroup of $G$ with three elements?  If the answer is yes, give the elements of the subgroup, and in any case give adequate explanation for your answers. 
  
  \item[(\textit{c})] Can you find an orbit with three elements?
  
  \item[(\textit{d})] Let $H$ be the stabilizer of $w$ in $G$. If we multiply $c \in G$ by every element of $H$--that is, find $ch$ for all $h \in H$--we get what is called a $\textit{left coset}$ of $H$, and it is denoted by $cH$. What are the elements of the left coset $cH$?
\end{itemize}
\end{problem}

\begin{solution}

\end{solution}

\end{document}
