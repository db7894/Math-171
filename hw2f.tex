% HMC Math dept HW template example
% v0.04 by Eric J. Malm, 10 Mar 2005
\documentclass[12pt,letterpaper,boxed]{hmcpset}

% set 1-inch margins in the document
\usepackage[margin=1in]{geometry}

% include this if you want to import graphics files with /includegraphics
\usepackage{graphicx}
\usepackage{amsmath}
\usepackage{afterpage}

\newcommand\blankpage{%
    \null
    \thispagestyle{empty}%
    \addtocounter{page}{-1}%
    \newpage}

% info for header block in upper right hand corner
\name{Daniel Bashir}
\class{Math 171}
\assignment{Homework 2f}
\duedate{01/26/2018}

\begin{document}

\problemlist{1.3.15, 1.4.4, 1.4.11, 1.5.6, 2.1.2}

\begin{problem}[1.3.15]
Let $ p $ be a prime, and $ G = (\mathbb{Z}/p\mathbb{Z})^{\times} $.
\begin{itemize}
  \item[(\textit{a})] Show that $ p-1 $ is its own inverse in $ G $.
  
  \item[(\textit{b})] Show that $ 1 $ and $ p-1 $ are the only elements of $ G $ that are their own inverses. 
  
  \item[(\textit{c})] (Wilson's Theorem.) Show that $ (p-1)! \equiv (p-1) $ mod $ p \equiv -1 $ mod $ p $.
\end{itemize}
\end{problem}

\begin{solution}

\end{solution}

\clearpage

\begin{problem}[1.4.4]
How many elements does GL(2,3) have? Justify your answer with an appeal to Theorem 1.64. Can you extend your argument to GL(2,$p$) where $p$ is an arbitrary prime?
\end{problem}

\begin{solution}
\end{solution}

\clearpage

\begin{problem}[1.4.11]
Define $$ \mathbb{Q}[\sqrt{3}] = \{a + b\sqrt{3} \vert a,b \in \mathbb{Q} \}.$$ Is $\mathbb{Q}[\sqrt{3}]$ a field? (The two operations are the usual addition and multiplication of numbers.)
\end{problem}

\begin{solution}
\end{solution}

\clearpage


\begin{problem}[1.5.6]
$ \textbf{The center of} $ GL($n$,$F$). Let $F$ be a field (the rationals, the reals, the complexes, or $ \mathbb{Z}/p\mathbb{Z} $). Show that $ \textbf{Z} $(GL($n$,$F$), the center of the general linear group, is $ \{\lambda I_n \vert \lambda \in F\} $, the set of matrices that are constant multiples of the identity matrix. You may find the following steps helpful: \\
$\textit{Step 1}$: For $ a \leq i, j \leq n $, let $E_{i,j}$ be the $n \times n$ matrix that has a one in the ($i$,$j$) entry and zeros elsewhere. Show that $B_{i,j} = I_n + E_{i,j}$ is an elementary matrix and hence an element of GL($n$,$F$). \\
$\textit{Step 2}$: Assume that $ A \in \textbf{A}$(GL($n$,$F$)). Show that $AB_{i,j} = B_{i,j}A$ implies that $AE_{i,j} = E_{i,j}A$. \\
$\textit{Step 3}$: If $ A \in \textbf{Z}$(GL($n$,$F$)), by comparing $ AE_{i,j} $ with $ E_{i,j}A $, conclude that $ A = \lambda I_n $ for some $ \lambda \in F $.
\end{problem}

\begin{solution}
\end{solution}

\clearpage

\begin{problem}[2.1.2]
Let $\theta$ be a real number and define $$ R_\theta = \begin{bmatrix} \cos(\theta) & -\sin(\theta) \\ \sin(\theta) & \cos(\theta) \end{bmatrix}. $$
\begin{itemize}
  \item[(\textit{a})] $R_\theta$ is called a rotation matrix. Can you explain why?
  
  \item[(\textit{b})] Show $$ R_\theta R_\mu = R_?, $$ $$ R_\theta^{-1} = R_?. $$
  
  \item[(\textit{c})] Let $$ G = \{R_\theta \vert \theta \in \mathbb{R} \}.$$ Show that $ G $ is a group under matrix multiplication. 
\end{itemize}
\end{problem}
\begin{solution}
\end{solution}
\end{document}
