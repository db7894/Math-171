\documentclass[12pt,letterpaper,boxed]{hmcpset}

\usepackage[margin=1in]{geometry}

\usepackage{graphicx}

\usepackage{enumitem}
\usepackage{amsmath}

\name{Daniel Bashir}
\class{Math 171}
\assignment{Homework 12f}
\duedate{04/13/2018}

\begin{document}

\problemlist{11.5.13, 12.2.8, 12.3.7, 12.3.12, 16.1.9}

\begin{problem}[11.5.13]
$\textbf{The Group}$ PSL(2,3). Recall that the special linear group SL($n$,$p$) is the group of $n \times n$ invertible matrices with determinant 1 over $\mathbb{F}_p = (\mathbb{Z}/p\mathbb{Z}, +, \cdot)$. The $\textit{Projective Special Linear Group}$, PSL($n$,$p$) is defined to be the quotient group $\text{SL}(n,p)/\textbf{Z}(\text{SL}(n,p))$. Find an already familiar group that is isomorphic to PSL(2,3).
\newline You may find the following steps helpful
\newline
$\textit{Step 1}$: If you have done Problem 9.3.1, then you may be able to guess the answer to this problem by looking at the relevant part of the lattice diagram for SL(2,3).
\newline
$\textit{Step 2}$; Let $\mathbb{F}_3 = (\mathbb{Z}/3\mathbb{Z},+,\cdot)$ be the field with three elements, and let $(\mathbb{F}_3)^2 = \{(a,b) \vert a,b \in \mathbb{F}_3 \}$. Show that $(\mathbb{F}_3)^2$ is a two dimensional vector space over $\mathbb{F}_3$.
\newline
$\textit{Step 3}$: Let $\Omega$ be the set of one dimensional subspaces of $(\mathbb{F}_3)^2$. What is $\vert \Omega \vert$?
\newline
$\textit{Step 4}$: Define a (natural) action of SL(2,3) on $\Omega$.
\newline
$\textit{Step 5}$: Show that the kernel of this action is $\textbf{Z}$(SL(2,3)).
\newline
$\textit{Step 6}$: Use Theorem 11.28 to identify PSL(2,3).
\end{problem}

\begin{solution}
\end{solution}

\clearpage

\begin{problem}[12.2.8]
Let $G$ be a group of order $168 = 2^3 \times 3 \times 7$. Assume that an element of order 7 is in the normalizer of a Sylow 2-subgroup of $G$. Prove that $G$ is not simple. 
\end{problem}

\begin{solution}
\end{solution}

\clearpage

\begin{problem}[12.3.7]
Our friend $G$ is a group of order 63.
\begin{enumerate}[label=\alph*]
\item The group $G$ is guaranteed to have subgroups of which sizes? Why?
\item Assume that $H$ is a subgroup of $G$ of order 21. Prove that $H \triangleleft G$.
\end{enumerate}
\end{problem}

\begin{solution}
\end{solution}

\clearpage

\begin{problem}[12.3.12]
Let $G$ be a group of order 30. Does $G$ have to have a subgroup of order 15? Does $G$ have to have an element of order 15?
\end{problem}

\begin{solution}
\end{solution}

\clearpage

\begin{problem}[16.1.9]
The elements of the set $\mathbb{Z}_{(5)}$ are rational numbers. A rational number $q$ is in the set $\mathbb{Z}_{(5)}$ if and only if $q$ can be written as $a/b$ where $a$ and $b$ are integers, gcd($a$,$b$) = 1, and $b$ is not divisible by 5. Is $\mathbb{Z}_{(5)}$ a ring? Is it a field? Can you find a non-trivial ideal of $\mathbb{Z}_{(5)}$? If the answer is yes, explicitly construct such an ideal, and if the answer is no, give a reason.
\end{problem}

\begin{solution}
\end{solution}

\end{document}