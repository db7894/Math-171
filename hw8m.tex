\documentclass[12pt,letterpaper,boxed]{hmcpset}

\usepackage[margin=1in]{geometry}

\usepackage{graphicx}

\usepackage{enumitem}

\name{Daniel Bashir}
\class{Math 171}
\assignment{Homework 8m}
\duedate{03/05/2018}

\begin{document}

\problemlist{5.4.8, 7.2.8, 7.3.1, 8.1.4, 8.2.2}

\begin{problem}[5.4.8]
Let $G$ be a finite group. Randomly and independently pick elements $x_1, x_2, x_3$ and $x_4$ from the group. (You are sampling with replacement. After picking $x_1$, return it to the group and choose another random element. Hence $x_2$ may equal $x_1$.) Assume that each of $ x_1,\cdots , x_4$ are elements of the enter of $G$. Based on this, you declare that $G$ is abelian. Could you be mistaken? Prove that, regardless of the size of $G$, the probability that you could be wrong is less than 0.004 (four-tenths of one percent). 
\end{problem}

\begin{solution}

\end{solution}

\clearpage

\begin{problem}[7.2.8]
Let $G$ be a group of order 21. Can $G$ be abelian and yet not cyclic? Prove your assertion. 
\end{problem}

\begin{solution}
\end{solution}

\clearpage

\begin{problem}[7.3.1]
$\textbf{Proof of Sylow D Theorem 7.13.}$ Using the outline in the text, prove that if $G$ is a finite group, $p$ a prime, $ P \in $ Syl$_p(G)$, $ Q \leq G$, and $ \vert Q \vert $ is a power of $p$, then $ Q \leq gPg^{-1} $ for some $g \in G$. 
\end{problem}

\begin{solution}
\end{solution}

\clearpage

\begin{problem}[8.1.4]
$\textbf{Jordan's Theorem.}$ Assume a finite group $G$ acts on a finite set $\Omega$. Further assume that $ | \Omega | >$ 1 and that the action is transitive (i.e., the action has only one orbit). Show that there exists $ g \in G $ with fix($g$) = 0. \newline
\hangindent=1cm An element $ g \in G $ with fix($g$) = 0 is called a $\textit{derangement}$.
\end{problem}

\begin{solution}
\end{solution}

\clearpage


\begin{problem}[8.2.2]
How many different patchwork quilts, four patches long and three patches wide, can be made from five red and seven blue squares, assuming that the quilts cannot be turned over?
\end{problem}

\begin{solution}
\end{solution}


\end{document}
