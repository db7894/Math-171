\documentclass[12pt,letterpaper,boxed]{hmcpset}

\usepackage[margin=1in]{geometry}

\usepackage{graphicx}

\usepackage{enumitem}
\usepackage{amsmath}

\name{Daniel Bashir}
\class{Math 171}
\assignment{Homework 10w}
\duedate{03/28/2018}

\begin{document}

\problemlist{9.2.7, 10.1.14, 11.4.2, 11.4.5, 11.4.8}


\begin{problem}[9.2.7]
Let $G = D_{10} \times \mathbb{Z}/7\mathbb{Z}$, where, as usual, $D_{10} - \langle a,b \vert a^5 = b^2 = e, ba = a^4b \rangle$. Let  $H = \langle (b,1) \rangle$ and $K = D_{10} \times \{0\}$.
\begin{enumerate}[label=\alph*]
\item Draw a partial lattice diagram of subgroups of $G$ that includes the subgroups $H$ and $K$. Include also $H \cap K$ and $\langle H,K \rangle$. Label the edges with appropriate numbers, and give reasons for what you have done
\newline In what follows, you may refer back to your diagram.
\item What is $\textbf{N}_G(H)$, the normalizer of $H$ in  $G$? Why?
\item Is $\textbf{N}_G(K) = G$?
\item What is $\vert \text{cl}_G(b,1) \vert$? Note that cl$_G(x)$ is the conjugacy class of $x$ in $G$ and that we are only interested in the order of this set.
\item Can you add the subgroup $L = \langle (a,0) \rangle$ to your lattice diagram?

\end{enumerate}
\end{problem}

\begin{solution}
\end{solution}

\clearpage

\begin{problem}[10.1.14]
Let  $G$ be a finite group, and let $N$ and $H$ be subgroups of $G$. Assume $N$ is normal in $G$ and gcd($\vert G:H \vert,\vert N \vert$) = 1. Show that $N \leq H$.

\end{problem}

\begin{solution}

\end{solution}

\clearpage

\begin{problem}[11.4.2]
According to Cayley's Theorem 11.35, every finite group is isomorphic to a subgroup of some $S_n$. Find the smallest $n$ such that $(\mathbb{Z}/6\mathbb{Z},+)$ is isomorphic to a subgroup of $S_n$. 
\end{problem}

\begin{solution}
\end{solution}

\clearpage

\begin{problem}[11.4.5]
Let $S_3$ act on $\Omega = S_3$ by conjugation, and let $\theta: S_3 \rightarrow S_6$ be the resulting homomorphism (see the previous problem). Label each element of $\Omega$ with 1,...,6, and explicitly give $\theta$((1 2 3)). Could you have done this problem more easily if you had used the Cayley diagraph of this action given in Figure 4.4? Use the Cayley diagraph to give $\theta$((1 2)).
\end{problem}

\begin{solution}
\end{solution}

\clearpage

\begin{problem}[11.4.8]
Identify the group of the rigid symmetries of a regular tetrahedron (i.e., find a well known group that is isomorphic to this group).
\end{problem}

\begin{solution}
\end{solution}

\clearpage

\end{document}