\documentclass[12pt,letterpaper,boxed]{hmcpset}

\usepackage[margin=1in]{geometry}

\usepackage{graphicx}

\usepackage{enumitem}
\usepackage{amsmath}

\name{Daniel Bashir}
\class{Math 171}
\assignment{Homework 14m}
\duedate{04/23/2018}

\begin{document}

\problemlist{16.1.14, 16.2.7, 18.1.4, 18.1.6, 18.1.8}



\begin{problem}[16.1.14]
Let $R = M_{2 \times 2}(\mathbb{R})$ be the ring of two-by-two matrices with real entries. Let $S = 
 \{ \begin{bmatrix}
    a & b \\
    0 & a
  \end{bmatrix}
  \vert a,b \in \mathbb{R} \} $ and $T = 
  \{ \begin{bmatrix}
    0 & b \\
    0 & 0
  \end{bmatrix}
  \vert b \in \mathbb{R} \} $.
\begin{enumerate}[label=\alph*]
\item Are $T$ and $S$ subrings of $R$?
\item Is $T$ an ideal of $S$? Is $T$ an ideal of $R$? Is $S$ an ideal of $R$?
\end{enumerate}
\end{problem}

\begin{solution}
\end{solution}

\clearpage

\begin{problem}[16.2.7]
Recall that the Gaussian integers are denoted by $\mathbb{Z}[i]$ and are defined by $$ \mathbb{Z}[i] = \{ a + bi \vert a,b \in  \mathbb{Z}, i^2 = -1 \}. $$
Let $I = \langle 1 + 3i \rangle$ be the ideal generated by $1+3i$ in $\mathbb{Z}[i]$, and define $R = \mathbb{Z}[i]/I$. 
\begin{enumerate}[label=\alph*]
\item $I+i$ and $I+3$ are t wo elements of $R$. Are they equal? What about $I+9$ and $I+1$?
\item How many elements does $R$ have?
\item Can you find a familiar ring that is isomorphic to $R$?
\end{enumerate}
\end{problem}

\begin{solution}
\end{solution}

\clearpage

\begin{problem}[18.1.4]
Find all the units, irreducibles, and primes in the ring $(\mathbb{Z}/10\mathbb{Z}, +, \cdot)$. 
\end{problem}

\begin{solution}
\end{solution}

\clearpage

\begin{problem}[18.1.6]
Is $\langle x^3 \rangle$ a prime ideal in $\mathbb{Z}[x]$? 
\end{problem}

\begin{solution}
\end{solution}

\clearpage

\begin{problem}[18.1.8]
Let $R$ be a commutative ring with identity. Is $x$ an irreducible element of $R[x]$? Either prove that it is or give a counterexample. If the answer is no, then give a condition on $R$ that would assure that $x$ is irreducible in $R[x]$.
\end{problem}

\begin{solution}
\end{solution}

\end{document}