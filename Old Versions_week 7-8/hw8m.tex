\documentclass[12pt,letterpaper,boxed]{hmcpset}

\usepackage[margin=1in]{geometry}

\usepackage{graphicx}

\usepackage{enumitem}

\name{Daniel Bashir}
\class{Math 171}
\assignment{Homework 8m}
\duedate{03/05/2018}

\begin{document}

\problemlist{5.4.8, 7.2.8, 7.3.1, 8.2.7, 9.2.6}

\begin{problem}[5.4.8]
Let $G$ be a finite group. Randomly and independently pick elements $x_1, x_2, x_3$ and $x_4$ from the group. (You are sampling with replacement. After picking $x_1$, return it to the group and choose another random element. Hence $x_2$ may equal $x_1$.) Assume that each of $ x_1,\cdots , x_4$ are elements of the enter of $G$. Based on this, you declare that $G$ is abelian. Could you be mistaken? Prove that, regardless of the size of $G$, the probability that you could be wrong is less than 0.004 (four-tenths of one percent). 
\end{problem}

\begin{solution}

\end{solution}

\clearpage

\begin{problem}[7.2.8]
Let $G$ be a group of order 21. Can $G$ be abelian and yet not cyclic? Prove your assertion. 
\end{problem}

\begin{solution}
\end{solution}

\clearpage

\begin{problem}[7.3.1]
$\textbf{Proof of Sylow D Theorem 7.13.}$ Using the outline in the text, prove that if $G$ is a finite group, $p$ a prime, $ P \in $ Syl$_p(G)$, $ Q \leq G$, and $ \vert Q \vert $ is a power of $p$, then $ Q \leq gPg^{-1} $ for some $g \in G$. 
\end{problem}

\begin{solution}
\end{solution}

\clearpage


\begin{problem}[8.2.7]
A certain kind of ceramic tile has the form of a 4 $\times$ 4 board. Each of the 16 squares is colored white, blue, or yellow. How many different such ceramic tiles are there? Note that we cannot turn the tiles upside down. 
\end{problem}

\begin{solution}
\end{solution}

\clearpage

\begin{problem}[9.2.6]
The group $G$ has 270 elements, and $Q$ is a subgroup of $G$ of order 9.  Assume $\textbf{N}_G(Q) = G$, and let $P$ be a Sylow 3-subgroup of $G$. What can you say about $\vert PQ \vert$ and $\vert P \cap Q \vert$?
\end{problem}

\begin{solution}

\end{solution}

\end{document}
