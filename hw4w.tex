\documentclass[12pt,letterpaper,boxed]{hmcpset}

\usepackage[margin=1in]{geometry}

\usepackage{graphicx}

\name{Daniel Bashir}
\class{Math 171}
\assignment{Homework 4w}
\duedate{02/07/2018}

\begin{document}

\problemlist{2.5.10, 2.6.22, 3.1.6, 3.2.4, 3.2.7}

\begin{problem}[2.5.10]
Let $m$ and $n$ be positive integers with gcd($m$,$n$) = 1, and let $\phi$ denote Euler's $\phi$-function (Definition 1.45). Consider the group $\mathbb{Z}/m\mathbb{Z} \times \mathbb{Z}/n\mathbb{Z}$.
\begin{itemize}
  \item[(\textit{a})] Show that ($a$,$b$) is a generator for the group $\mathbb{Z}/m\mathbb{Z} \times \mathbb{Z}/n\mathbb{Z}$ if and only if $a$ and $b$ are, respectively, generators for $\mathbb{Z}/m\mathbb{Z} $ and $ \mathbb{Z}/n\mathbb{Z}$. 
  
  \item[(\textit{b})] Show that the number of generators of the group $\mathbb{Z}/m\mathbb{Z} \times \mathbb{Z}/n\mathbb{Z}$ is $\phi$($n$)$\phi$($m$).
  
  \item[(\textit{c})] Prove that for relatively prime integers $m$ and $n$, $$\phi(mn) = \phi(m)\phi(n).$$
\end{itemize}
\end{problem}

\begin{solution}

\end{solution}

\clearpage

\begin{problem}[2.6.22]
Let $G$ be a cyclic group of order $n$. Let $m \leq n$ be a positive integer. How many subgroups of order $m$ does $G$ have? Prove your assertion.
\end{problem}

\begin{solution}
\end{solution}

\clearpage

\begin{problem}[3.1.6]
Let $x$ and $y$ be two 3-cycles. Can $xy$ be a 3-cycle? A 5-cycle? An element of order 2? In each case either give an example, or prove that it is impossible. 
\end{problem}

\begin{solution}
\end{solution}

\clearpage


\begin{problem}[3.2.4]
How many elements of order 2 does $A_5$ have?
\end{problem}

\begin{solution}
\end{solution}

\clearpage

\begin{problem}[3.2.7]
Is $A_4$ isomorphic to $D_{12}$? Why?
\end{problem}

\begin{solution}

\end{solution}

\end{document}
