\documentclass[12pt,letterpaper,boxed]{hmcpset}

\usepackage[margin=1in]{geometry}

\usepackage{graphicx}

\usepackage{enumitem}

\setlength{\parindent}{4em}

\name{Daniel Bashir}
\class{Math 171}
\assignment{Homework 7f}
\duedate{02/30/2018}

\begin{document}

\problemlist{4.5.1, 5.2.15, 7.2.6, 8.1.3, 8.2.1}

\begin{problem}[4.5.1]
$ \textbf{A simple card trick} $ \newline 
\hangindent=1cm
\hangafter=1
The magician has a spectator choose a card, memorize it, and return it to the top of the deck. She then allows a spectator to cut the cards--split the deck into two by taking a set of cards from the top, and then switch the two parts--as many times as he would like. The magician spreads the cards face up and announces the chosen card. \newline

Analyze and explain the above trick using the following steps: \newline
$\textit{Step 1}$: Consider a deck of 52 cards $C_1,C_2,...,C_{52}$, and let $H$ be the subgroup of $S_{52}$ generated by (1 2 $\cdots$ 51 52). If $ \sigma \in H $, then define $ \sigma \cdot C_i = C_{\sigma(i)} $. Show that this gives an action of $H$ on the deck of cards. \newline
$\textit{Step 2}$: Let $ \tau = ( 1 \ 2 \ \cdots \ 51 \ 52 ) $ and put the deck in the order $ C_1,C_2,...,C_{52} $. Now, if we apply $\tau$ to the deck of cards, then what happens to the order of the cards? What if we apply $\tau^2$? Show that any "cutting of the cards" can be achieved by the action of an element of $H$. \newline
$\textit{Step 3}$: Let $\Omega = \{ \{ C_1,C_2 \}, \{ C_2,C_3 \},..., \{ C_{52}, C_1 \} \} $ be the set of consecutive pairs of cards in the original deck. Show that the action of $H$ on the deck of cards results in an action of $H$ on $\Omega$. Conclude that cutting the deck does not change the set of consecutive pairs of cards. \newline
$\textit{Step 4}$: Explain the card trick.
\end{problem}

\begin{solution}

\end{solution}

\clearpage

\begin{problem}[5.2.15]
$\textbf{Proof of Theorem 5.24(c).}$ Let $ U,V \leq G $ with $ \vert G \ : \ U \vert $ and $ \vert G \ : \ V \vert < \infty $. Assume gcd($\vert G \ : \ U \vert $, $\vert G \ : \ V \vert $) = 1. Show $ G = VU $. 

\end{problem}

\begin{solution}
\end{solution}

\clearpage

\begin{problem}[7.2.6]
Let $G$ be a group of order $ 539 = 7^2 \times 11 $. Assume that $G$ acts on a set with ten elements and that there is some orbit of size bigger than 1. 
\begin{enumerate}[label=\alph*]
\item What can you say about the orbit sizes? Why?
\item $G$ is guaranteed to have subgroups of which sizes? Give reasons or proofs for your assertions. 
\end{enumerate}
\end{problem}

\begin{solution}
\end{solution}

\clearpage


\begin{problem}[8.1.3]
Let $n$ be a fixed positive integer. We randomly pick an element $ x \in S_n $, and we count the number of fixed points of $x$ (e.g., if $ x = (1 \ 2 \ 5) \in S_5 $, then $x$ fixes 3 and 4 and hence we record two fixed points). What is the expected value of the number of fixed points?
\end{problem}

\begin{solution}
\end{solution}

\clearpage

\begin{problem}[8.2.1]
Compute how many different ways there are to color the faces of a cube so that three faces are red, two are white, and one is blue. 
\end{problem}

\begin{solution}

\end{solution}

\end{document}