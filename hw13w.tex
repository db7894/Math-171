\documentclass[12pt,letterpaper,boxed]{hmcpset}

\usepackage[margin=1in]{geometry}

\usepackage{graphicx}

\usepackage{enumitem}
\usepackage{amsmath}

\name{Daniel Bashir}
\class{Math 171}
\assignment{Homework 13w}
\duedate{04/18/2018}

\begin{document}

\problemlist{12.3.23, 16.2.12, (16.3.1, 16.3.2), 16.3.7, 17.1.1}

\begin{problem}[12.3.23]
Let $G$ be a group of order 180. Prove that $G$ is not simple.
\newline If you assume that $G$ $\textit{is}$ simple, then you may find the following steps helpful.
\newline
Step 1: Show that the number of Sylow 3-subgroups of $G$ cannot be 4.  Conclude that the number of Sylow 3-subgroups of $G$ must be 10.
\newline
Step 2: Show that if the number of Sylow 5-subgroups of $G$ is 36, then at least two  of the Sylow 3-subgroups must intersect non-trivially.
\newline
Step 3: Let $P$ and $Q$ be two Sylow 3-subgroups of $G$, and assume that they intersect non-trivially. Define $R = P \cap Q$, and $L = \rangle P,Q \langle$. Show that $R \triangleleft L$, and so $L < G$. Further (by drawing a partial lattice diagram and using Theorem 9.27), show that $\vert L \vert \geq 36$.
\newline
Step 4: Use the subgroup $L$ in the previous part to show that the number of Sylow 5-subgroups of $G$ cannot be 36. Conclude that the number of Sylow 5-subgroups of $G$ must be 6.
\newline
Step 5: Do Problem 12.3.12, and apply your results to the normalizer of the Sylow 5-subgroup.
\newline
Step 6: Show that $G$ must be isomorphic to a subgroup of $S_6$.
\newline
Step 7: Arrive at a contradiction.
\end{problem}

\begin{solution}
\end{solution}

\clearpage

\begin{problem}[16.2.12]
Let $I$ be an ideal in a commutative ring $R$. Prove that $I[x]$ is an ideal in $R[x]$. Prove that $R[x]/I[x] \cong (R/I)[x]$.
\end{problem}

\begin{solution}
\end{solution}

\clearpage

\begin{problem}[16.3.1]
Can you find a commutative ring with identity with characteristic 5 that is not an integral domain? Either give an example or prove that it is not possible.
\end{problem}

\begin{solution}
\end{solution}

\clearpage

\begin{problem}[16.3.2]
Is it possible to find an infinite ring with characteristic 3? Either give an example or prove that it is not possible.
\end{problem}

\begin{solution}
\end{solution}

\clearpage

\begin{problem}[16.3.7]
Let $R = \mathbb{Z}[i]$ be the ring of Gaussian integers, and let $I = \langle 2 + 3i \rangle$. What is the characteristic of the ring $R/I$?
\end{problem}

\begin{solution}
\end{solution}

\clearpage

\begin{problem}[17.1.1]
What is the field of fractions of the Gaussian integers $\mathbb{Z}[i] = \{ a + bi \ \vert \ a,b \in \mathbb{Z} \}$?
\end{problem}

\begin{solution}

\end{solution}

\end{document}