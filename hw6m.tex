\documentclass[12pt,letterpaper,boxed]{hmcpset}

\usepackage[margin=1in]{geometry}

\usepackage{graphicx}

\name{}
\class{Math 171}
\assignment{Homework 6m}
\duedate{02/19/2018}

\begin{document}

\problemlist{3.3.5, 5.1.17, 5.2.5, 5.2.12, 5.3.2}

\begin{problem}[3.3.5]
To analyze the permutation puzzle described here, consider the placement of names in the briefcases as a permutation of 1,...,100. In other words, the initial placement of the names in the briefcases is an element of $S_{100}$. Prove that the suggested strategy will succeed if this element of $S_{100}$ has no cycle of size more than 50. Using your answer to Problem 3.3.4, prove that if the contestants agree on the suggested strategy, then the probability of their success (i.e., $\textit{all}$ of them finding the right briefcase) is more than 30\%. 
\end{problem}

\begin{solution}

\end{solution}

\clearpage

\begin{problem}[5.1.17]
Let $ G $ be a group, and let $ H \leq G $. Assume that the number of elements in $H$ is half of the number of elements in $G$. 
\begin{itemize}
  \item[(\textit{a})] How many right cosets does $H$ have in $G$?
  
  \item[(\textit{b})] If $x \in G$, show that $x^2 \in H$. 
  
  \item[(\textit{c})] Let $g \in G$ with $o(g) = 3$. Show that $g \in H$.
\end{itemize}
\end{problem}

\begin{solution}
\end{solution}

\clearpage

\begin{problem}[5.2.5]
Let $D_{10} = \langle a,b \vert a^5 = b^2 = e, ba = a^4b \rangle$ be the dihedral group of order 10. Assume $x$ and $y$ are two distinct elements of order 2 in $D_{10}$. Let $H = \langle x,y \rangle$. What can you say about $\vert H \vert$? Can $x$ and $y$ commute? Give your reasons. 
\end{problem}

\begin{solution}
\end{solution}

\clearpage


\begin{problem}[5.2.12]
Give two examples of $U,V \leq G$ with $\vert G \ : \ U \vert < \infty$. Each example should satisfy one of the following: 
\begin{itemize}
  \item[(\textit{a})] $\vert V \ : \ V \cap U \vert < \vert G \ : \ U \vert$.
  
  \item[(\textit{b})] $\vert V \ : \ V \cap U \vert = \vert G \ : \ U \vert$.
\end{itemize}
\end{problem}

\begin{solution}
\end{solution}

\clearpage

\begin{problem}[5.3.2]
Let $a$ be a positive integer. Prove that $a^{21}$ and $a$ have the same remainder when divided by 15. 
\end{problem}

\begin{solution}

\end{solution}

\end{document}
