\documentclass[12pt,letterpaper,boxed]{hmcpset}

\usepackage[margin=1in]{geometry}

\usepackage{graphicx}

\name{}
\class{Math 171}
\assignment{Homework 6w}
\duedate{02/21/2018}

\begin{document}

\problemlist{5.1.10, 5.1.19, 5.2.13, 6.1.3, 6.1.4}

\begin{problem}[5.1.10]
Let $G$ be a group, and let $H \leq G$. Recall definition 4.24 of $\textbf{N}_G(H)$, the normalizer of $H$ in $G$. Show that
$$ \textbf{N}_G(H) = \{ x \in G \vert xHx^{-1} = H \} = \{ x \in G \vert xH = Hx \}. $$
\end{problem}

\begin{solution}

\end{solution}

\clearpage

\begin{problem}[5.1.19]
Let $G = A_4$, the alternating group of degree 4.
\begin{itemize}
  \item[(\textit{a})] How many elements of order 3 does $G$ have?
  
  \item[(\textit{b})] Does $G$ have a subgroup of order 6?
\end{itemize}
\end{problem}

\begin{solution}
\end{solution}

\clearpage

\begin{problem}[5.2.13]
$\textbf{Proof of Theorem 5.24(a).}$ Let $ U,V \leq G $ with $ |G \ : \ U| < \infty $. Show that we have $ |V \ : \ V \cap U| \leq |G \ : \ U| $.
\end{problem}

\begin{solution}
\end{solution}

\clearpage


\begin{problem}[6.1.3]
Let $G$ be a group of order 121, and let $\Omega$ be a set with 16 elements. Assume that $G$ acts on $\Omega$. We are given that there is some orbit of size bigger than one. Let $\Omega_0$ be the set of elements in $\Omega$ that are fixed by every element of $G$. What can you say about $\vert \Omega_0 \vert$? Prove your assertions. 
\end{problem}

\begin{solution}
\end{solution}

\clearpage

\begin{problem}[6.1.4]
Recall Definition 4.14 of the conjugation action of a group on the set of its subgroups as well as Definition 4.24 of normalizers. Let $G = D_8$ act on the set of all of its subgroups by conjugation. 
\begin{itemize}
  \item[(\textit{a})] Let $H = \langle b \rangle$. In problem $\textit{4.3.3}$, you found $\textbf{N}_G(H)$. Use the FCP to find the size of the orbit of $H$. What are the elements in the orbit of $H$?
  
  \item[(\textit{b})] Let $K = \langle a \rangle$. Find $\textbf{N}_G(K)$, and the orbit of $K$. 
\end{itemize}
\end{problem}

\begin{solution}

\end{solution}

\end{document}
