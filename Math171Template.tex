\documentclass[11pt]{article}
\usepackage{geometry}
\usepackage{amsmath}
\usepackage{amssymb}
\usepackage{enumitem}
\usepackage{fancyhdr}
\usepackage{graphicx}
\usepackage{lastpage}
\usepackage{parskip}

\newif\ifclearpage

\newcommand{\name}{}
\newcommand{\class}{Math 171, Spring 2017}
\newcommand{\assignment}{Problem Set 1}
\newcommand{\duedate}{Due Friday, January 19}

\clearpagetrue

\newcommand{\problem}[1]{\section*{#1}}
\newcommand{\solution}{\hrulefill}
\newcommand{\maybeclearpage}{\ifclearpage\clearpage\fi}

\renewcommand{\vec}[1]{\mathbf{#1}}

\geometry{margin=1in}

\fancypagestyle{primary}{
  \fancyhf{}
  \lhead{\name}
  \chead{\assignment}
  \rhead{\class}
  \lfoot{\duedate}
  \rfoot{\thepage{} of \pageref{LastPage}}}

\pagestyle{primary}

\setlist[enumerate]{label=(\alph*)}

\begin{document}

\problem{Shahriari Problem 1.1.1}

Determine the probability that a photon is detected at the first
minimum of a six-slit grating if the bottom two slits are closed.
Assume the magnitude of the probability amplitude due to each slit is
$r$. \emph{Suggestion:} Start by showing how the complex probability
amplitudes from each slit add up to zero at the first minimum.

\solution



\maybeclearpage
\problem{Shahriari Problem 1.1.5}

Use the principle of least time to derive Snell's law, namely, $n_1
\sin \theta_1 = n_2 \sin \theta_2$ for light being refracted as it
travels from a medium with index of refraction $n_1$ into a medium
with index of refraction $n_2$. \emph{Suggestion:} Follow a procedure
similar to the one given in Example~1.11. Locate the source S in
medium 1 and the point P in medium 2.

\solution



\maybeclearpage
\problem{Shahriari Problem 1.1.6}

Use the principle of least time to derive Snell's law, namely, $n_1
\sin \theta_1 = n_2 \sin \theta_2$ for light being refracted as it
travels from a medium with index of refraction $n_1$ into a medium
with index of refraction $n_2$. \emph{Suggestion:} Follow a procedure
similar to the one given in Example~1.11. Locate the source S in
medium 1 and the point P in medium 2.

\solution


\maybeclearpage
\problem{Shahriari Problem 1.2.1}

Use the principle of least time to derive Snell's law, namely, $n_1
\sin \theta_1 = n_2 \sin \theta_2$ for light being refracted as it
travels from a medium with index of refraction $n_1$ into a medium
with index of refraction $n_2$. \emph{Suggestion:} Follow a procedure
similar to the one given in Example~1.11. Locate the source S in
medium 1 and the point P in medium 2.

\solution


\maybeclearpage
\problem{Shahriari Problem 1.2.11}

Use the principle of least time to derive Snell's law, namely, $n_1
\sin \theta_1 = n_2 \sin \theta_2$ for light being refracted as it
travels from a medium with index of refraction $n_1$ into a medium
with index of refraction $n_2$. \emph{Suggestion:} Follow a procedure
similar to the one given in Example~1.11. Locate the source S in
medium 1 and the point P in medium 2.

\solution

\end{document}
