\documentclass[12pt,letterpaper,boxed]{hmcpset}

\usepackage[margin=1in]{geometry}

\usepackage{graphicx}

\usepackage{enumitem}
\usepackage{amsmath}

\name{Daniel Bashir}
\class{Math 171}
\assignment{Homework 8w}
\duedate{03/07/2018}

\begin{document}

\problemlist{7.3.3, 9.1.3 8.2.7, 9.2.6, 10.1.2}

\begin{problem}[7.3.3]
$ \textbf{Proof of Corollary 7.15.} $ Let $G$ be a finite group and $p$ a prime. By completing the outline in the text, write a complete proof of the fact that Syl$_p(G)$ is exactly one orbit of the conjugation action of $G$ on the set of subgroups of $G$, and that if $P \in$ Syl$_p(G)$, then $$\vert Syl_p(G) \vert = \vert G \ : \ \textbf{N}_G(P) \vert. $$ Conclude that the number of Sylow $p$-subgroups is a divisor of the order of the group. 
\end{problem}

\begin{solution}

\end{solution}

\clearpage

\begin{problem}[9.1.3]
\begin{enumerate}[label=\alph*]
\item Draw the lattice of subgroups of $ \mathbb{Z}/6\mathbb{Z} $.
\item Repeat the above for the group of $S_3$.
\end{enumerate}
\end{problem}

\clearpage


\begin{problem}[8.2.7]
A certain kind of ceramic tile has the form of a 4 $\times$ 4 board. Each of the 16 squares is colored white, blue, or yellow. How many different such ceramic tiles are there? Note that we cannot turn the tiles upside down. 
\end{problem}

\begin{solution}
\end{solution}

\clearpage

\begin{problem}[9.2.6]
The group $G$ has 270 elements, and $Q$ is a subgroup of $G$ of order 9.  Assume $\textbf{N}_G(Q) = G$, and let $P$ be a Sylow 3-subgroup of $G$. What can you say about $\vert PQ \vert$ and $\vert P \cap Q \vert$?
\end{problem}

\begin{solution}

\end{solution}

\clearpage

\begin{problem}[10.1.2]
Let $ D_8 = \langle a,b \ \vert \ a^4 = b^2 = e, ba = a^3b \rangle $, and let $S_3$ be the symmetric group of degree 3. Let $ G = D_8 \times S_3 $. Let $ H = \langle b \rangle \times \langle (1 \ 2 \ 3) \rangle $ and $ K = \langle a \rangle \times \langle (1 \ 2 \ 3) \rangle $ be subgroups of $G$. Is $H$ a normal subgroup of $G$? What about $K$?
\end{problem}

\begin{solution}
\end{solution}

\end{document}
