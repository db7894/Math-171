\documentclass[12pt,letterpaper,boxed]{hmcpset}

\usepackage[margin=1in]{geometry}

\usepackage{graphicx}

\name{Daniel Bashir}
\class{Math 171}
\assignment{Homework 4f}
\duedate{02/09/2018}

\begin{document}

\problemlist{2.6.27, 2.6.32, 3.1.9, 4.1.7, 4.2.4}

\begin{problem}[2.6.27]
$\textbf{Conjugate Subgroups.}$ Let $G$ be a group. Let $H \leq G$, and let $x \in G$. We use the notation $xHx^{-1}$ to denote the set of elements $\{ xhx^{-1} \vert h \in H \}$. ($xHx^{-1}$ is called a $\textit{conjugate}$ of $H$.)
\begin{itemize}
  \item[(\textit{a})] Prove that $xHx^{-1}$ is a subgroup of $G$.
  
  \item[(\textit{b})] If $H$ is finite, then how are $|H|$ and $|xHx^{-1}|$ related?
  
  \item[(\textit{c})] Prove that $H$ is isomorphic to $xHx^{-1}$.
\end{itemize}
\end{problem}

\begin{solution}

\end{solution}

\clearpage

\begin{problem}[2.6.32]
\begin{itemize}
  \item[(\textit{a})] If $G$ is abelian and $A$ and $B$ are subgroups of $G$, prove that $AB$ is a subgroup of $G$.
  
  \item[(\textit{b})] Give an example of a group $G$ and two subgroups $A$ and $B$ of $G$, such that $AB$ is not a subgroup of $G$.
\end{itemize}
\end{problem}

\begin{solution}
\end{solution}

\clearpage

\begin{problem}[3.1.9]
$\textbf{Conjugate elements.}$ Let $\sigma$, $\tau \in S_n$. Define $\delta = \tau \sigma \tau^{-1}$. ($\sigma$ and $\delta$ are called $\textit{conjugate elements}$; see Chapter 6.)
\begin{itemize}
  \item[(\textit{a})] Show that if $\sigma$($i$) = $j$, then $\delta(\tau(i)) = \tau(j)$.
  
  \item[(\textit{b})] Explain that the previous part says that if you apply $\tau$ to each of the $\textit{entries}$ in the cycle notation of $\sigma$, then you get the cycle notation for $\delta$. In other words, if $\sigma$ has cycle decomposition $$(a_1 \ a_2 \cdots \ a_{k_1})(b_1 \ b_2 \ \cdots \ b_{k_2}) \cdots .$$ Then $\delta$ has cycle decomposition $$(\tau(a_1) \ \tau(a_2) \ \cdots \ \tau(a_{k_1}))(\tau(b_1) \ \tau(b_2) \ \cdots \ \tau(b_{k_2})) \cdots.$$
  
  \item[(\textit{c})] Illustrate the previous part, by letting $\sigma$ = (1 4 3 8)(2 6 5), $\tau$ = (1 6 3)(7 5 2), and quickly writing down the cycle decomposition for $\tau \sigma \tau^{-1}$. Check your answer by actually finding the product $\tau \sigma \tau^{-1}$.
\end{itemize}
\end{problem}

\begin{solution}
\end{solution}

\clearpage


\begin{problem}[4.1.7]
Let $G = D_8$ act on $\Omega$ = $D_8$ by conjugation. Make an 8 $\times$ 8 table, where the rows are indexed by the elements of $G$ and the columns are indexed by the elements of $\Omega$. The entry in row $g$ and column $x$ is the result of the action of $g$ on $x$ (that is $g \cdot x$.) Complete the table. 
\begin{itemize}
  \item[(\textit{a})] In each row do you get every element of $\Omega$? Does any element of $\Omega$ ever appear in any row more than once? Is this a coincidence?
  
  \item[(\textit{b})] Does an element of $\Omega$ ever appear in any column more than once? Is it possible for a column to contain only one element of $\Omega$? When does this happen?
  
  \item[(\textit{c})] Can two rows be identical?
  
  \item[(\textit{d})] Is it possible to tell the order of the elements of the group from this table?
\end{itemize}
\end{problem}

\begin{solution}
\end{solution}

\clearpage

\begin{problem}[4.2.4]
Let $G = D_8 = \langle a,b \vert a^4 = b^2 = e, ba = a^3b \rangle$ act by conjugation o $\Omega = D_8$. Let $S = \{ a,b \}$ be a set of generators for $D_8$. Draw the Cayley diagraph of the action for this set of generators. 
\end{problem}

\begin{solution}

\end{solution}

\end{document}
