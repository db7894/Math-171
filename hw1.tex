\documentclass[12pt,letterpaper,boxed]{hmcpset}

\usepackage[margin=1in]{geometry}

\usepackage{graphicx}

\name{Your Name}
\class{Math 171}
\assignment{Homework 1f}
\duedate{01/19/2018}

\begin{document}

\problemlist{1.1.1, 1.1.5, 1.1.6, 1.2.1, 1.2.11}

\begin{problem}[1.1.1]
Complete the multiplication table for $ D_8 $. 
\end{problem}

\begin{solution}

\end{solution}

\clearpage

\begin{problem}[1.1.5]
\begin{itemize}
  \item[(\textit{a})] Find the center of $ D_8 $.
  
  \item[(\textit{d})] Find $ \textbf{C}_{D_8} (R_{90}) $ and $ \textbf{C}_{D_8} (H) $.
\end{itemize}
\end{problem}

\begin{solution}
\end{solution}

\clearpage

\begin{problem}[1.1.6]
Let $ D_6 $ denote the set of symmetries of an equilateral triangle. Find the multiplication table for $ D_6 $. What is the center of $ D_6 $?
\end{problem}

\begin{solution}
\end{solution}

\clearpage


\begin{problem}[1.2.1]
Let $ \Omega = \mathbb{Z} $ be the set of integers. Define $ f: \Omega \to \Omega $ by $ f(x) = x + 5 $. Is $ f \in $ Perm($ \Omega $) ? If so, what is its inverse? If $ n $ is a positive integer, then what is $ f^{n}(x) $? What if instead of $ \mathbb{Z} $, we had $ \Omega = \mathbb{Z}^{\geq(0)} $, the set of non-negative integers? 
\end{problem}

\begin{solution}
\end{solution}

\clearpage

\begin{problem}[1.2.11]
In an algebra book you read the following definition: The function $ g: Y \to X $ is the inverse of the function $ f: X \to Y $ if the two diagrams in Figure 1.9 commute. 
Is this any different from our definition of inverses? Can you draw one diagram--with four nodes and five arrows--that is commutative if and only if $ g $ is the inverse of $ f $?
\end{problem}

\begin{solution}

\end{solution}

\end{document}
