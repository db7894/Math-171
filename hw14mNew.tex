\documentclass[12pt,letterpaper,boxed]{hmcpset}

\usepackage[margin=1in]{geometry}

\usepackage{graphicx}

\usepackage{enumitem}
\usepackage{amsmath}

\name{Daniel Bashir}
\class{Math 171}
\assignment{Homework 14m}
\duedate{04/23/2018}

\begin{document}

\problemlist{17.1.14, 18.1.8, 18.1.13, 18.1.16, 18.1.18}


\begin{problem}[17.1.14]
Let $R = \mathbb{Z}$, and let $M = \{1,5,5^2,5^3,...\}$. Let $S = R[M^{-1}]$ be the localization of $R$ at $M$. Let $I$ and $J$ be the ideals generated by 7 in $R$ and $S$, respectively. What are the elements of $J$ and how is $J \cap R$ related to $I$? Answer the same questions if we replaced 7 by 5.
\end{problem}

\begin{solution}
\end{solution}

\clearpage

\begin{problem}[18.1.8]
Let $R$ be a commutative ring with identity. Is $x$ an irreducible element of $R[x]$? Either prove that it is or give a counterexample. If the answer is no, then give a condition on $R$ that would assure that $x$ is irreducible in $R[x]$.
\end{problem}

\begin{solution}
\end{solution}

\clearpage

\begin{problem}[18.1.13]
Let $R$ be a commutative ring with identity.
\begin{enumerate}[label=\alph*]
\item Show that $R[x] / \langle x \rangle \cong R$.
\item Assume $R[x]$ is a PID. Show that $R$ is a field.
\end{enumerate}
\end{problem}

\begin{solution}
\end{solution}

\clearpage

\begin{problem}[18.1.16]
Let $R = \mathbb{C}[x,y]$ be the ring of polynomials in two variables over $\mathbb{C}$. Let $P = \langle x \rangle$ be the ideal of $R$ generated by $x$. Define a map $\theta: R \rightarrow \mathbb{C}[y]$ by $\theta(p(x,y)) = p(0,y)$. In other words, given a polynomial in two variables $x$ and $y$, we plug in 0 for $x$ to get a polynomial in $y$.
\begin{enumerate}[label=\alph*]
\item Is $\theta$ a ring homomorphism? What is the kernel and the image eof  $\theta$?
\item Is $P$ a prime ideal? Is $P$ a maximal ideal?
\item Is $\mathbb{C}[x,y]$ a PID?
\end{enumerate}
\end{problem}

\begin{solution}
\end{solution}

\clearpage

\begin{problem}[18.1.18]
Show that 5 is not irreducible in $\mathbb{Z}[i]$, the ring of Gaussian integers.
\end{problem}

\begin{solution}
\end{solution}

\end{document}